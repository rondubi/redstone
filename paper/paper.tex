%%%%%%%%%%%%%%%%%%%%%%%%%%%%%%%%%%%%%%%%%%%%%%%%%%%%%%%%%%%%%%%%%%%%%%%%%%%%%%%%
% Template for USENIX papers.
%
% History:
%
% - TEMPLATE for Usenix papers, specifically to meet requirements of
%   USENIX '05. originally a template for producing IEEE-format
%   articles using LaTeX. written by Matthew Ward, CS Department,
%   Worcester Polytechnic Institute. adapted by David Beazley for his
%   excellent SWIG paper in Proceedings, Tcl 96. turned into a
%   smartass generic template by De Clarke, with thanks to both the
%   above pioneers. Use at your own risk. Complaints to /dev/null.
%   Make it two column with no page numbering, default is 10 point.
%
% - Munged by Fred Douglis <douglis@research.att.com> 10/97 to
%   separate the .sty file from the LaTeX source template, so that
%   people can more easily include the .sty file into an existing
%   document. Also changed to more closely follow the style guidelines
%   as represented by the Word sample file.
%
% - Note that since 2010, USENIX does not require endnotes. If you
%   want foot of page notes, don't include the endnotes package in the
%   usepackage command, below.
% - This version uses the latex2e styles, not the very ancient 2.09
%   stuff.
%
% - Updated July 2018: Text block size changed from 6.5" to 7"
%
% - Updated Dec 2018 for ATC'19:
%
%   * Revised text to pass HotCRP's auto-formatting check, with
%     hotcrp.settings.submission_form.body_font_size=10pt, and
%     hotcrp.settings.submission_form.line_height=12pt
%
%   * Switched from \endnote-s to \footnote-s to match Usenix's policy.
%
%   * \section* => \begin{abstract} ... \end{abstract}
%
%   * Make template self-contained in terms of bibtex entires, to allow
%     this file to be compiled. (And changing refs style to 'plain'.)
%
%   * Make template self-contained in terms of figures, to
%     allow this file to be compiled. 
%
%   * Added packages for hyperref, embedding fonts, and improving
%     appearance.
%   
%   * Removed outdated text.
%
%%%%%%%%%%%%%%%%%%%%%%%%%%%%%%%%%%%%%%%%%%%%%%%%%%%%%%%%%%%%%%%%%%%%%%%%%%%%%%%%

\documentclass[letterpaper,twocolumn]{article}

\usepackage{style}

% to be able to draw some self-contained figs
\usepackage{tikz}
\usepackage{amsmath}

% inlined bib file
\usepackage{filecontents}

\usepackage{setspace}
\linespread{1.25}

%-------------------------------------------------------------------------------
\begin{filecontents}{\jobname.bib}
%-------------------------------------------------------------------------------
@Book{arpachiDusseau18:osbook,
author =       {Arpaci-Dusseau, Remzi H. and Arpaci-Dusseau Andrea C.},
title =        {Operating Systems: Three Easy Pieces},
publisher =    {Arpaci-Dusseau Books, LLC},
year =         2015,
edition =      {1.00},
note =         {\url{http://pages.cs.wisc.edu/~remzi/OSTEP/}}
}
@InProceedings{waldspurger02,
author =       {Waldspurger, Carl A.},
title =        {Memory resource management in {VMware ESX} server},
booktitle =    {USENIX Symposium on Operating System Design and
Implementation (OSDI)},
year =         2002,
pages =        {181--194},
note =         {\url{https://www.usenix.org/legacy/event/osdi02/tech/waldspurger/waldspurger.pdf}}}
\end{filecontents}

%-------------------------------------------------------------------------------
\begin{document}
%-------------------------------------------------------------------------------

%don't want date printed
\date{}

% make title bold and 14 pt font (Latex default is non-bold, 16 pt)
\title{\Large \bf Redstone: a Modular Deterministic Simulation Testing Framework for Distributed Systems}

%for single author (just remove % characters)
\author{
{\rm Benjamin Friedman}\\
% Your Institution
\and
{\rm Katherine Li}\\
% Second Institution
\and
{\rm Matthew Mattei}\\
\and
{\rm Ron Dubinsky}\\
% copy the following lines to add more authors
% \and
% {\rm Name}\\
%Name Institution
} % end author

\maketitle

%-------------------------------------------------------------------------------
\begin{abstract}
%-------------------------------------------------------------------------------
{\fontsize{12}{15}\selectfont 
In testing distributed systems,
confirming a system's tolerance of faults in disks and networks is not only useful but necessary.
Redstone, a modular deterministic simulation testing framework,
amplifies failures in various substrates underlying a distributed system
in order to confirm the system's ability to withstand these failures.
We intercept calls to the kernel for the disk and the network, as well as to clock functions,
introducing our own code.
Determinism allows reproduction of failure conditions,
allowing developers to confirm, having encountered and debugged a failure,
that their system will not fail again when confronted by identical conditions.
Our control over clock functions further allows us to "accelerate" time,
increasing ability to test faster, and to emulate the effects of clock jitter.
Finally, because Redstone is inherently modular,
it is simple to extend and port to a wider variety of applications and platforms.
}
\end{abstract}


%-------------------------------------------------------------------------------
\section{Introduction}
%-------------------------------------------------------------------------------

{\fontsize{12}{15}\selectfont 
Distributed systems are complicated ones,
because they rely not only on the particular machines
on which the systems run but also on the network connecting these machines.
As such, distributed systems must contain many
(often complicated and difficult-to-reason-about)
mechanisms by which to recover from both network and machine failures
(more specifically, network latency, disk I/O failure, bugs in the kernel,
and all manners of timing-based bugs).

Testing offers the ability to confirm that a system can withstand failures.
Specifically, if one can introduce a failure and check that the system withstands it,
it certainly increases the degree of trust in the system.
However, in order to be effective,
such testing must be able to consistently introduce failures.
Just hoping that failures naturally occur in the process of testing is a non-starter;
they must be introduced.
In order to introduce failures, the system must have control of the network, disk,
and time systems on which a distributed system relies.
Therefore, the system must simulate this functionality whilst also injecting failure where possible.

"Consistently" is another key phrase here,
as a test must be replicable.
% just running the system a few times doesn't confirm that everything that can go wrong has gone wrong.
As such,
a good testing system must not only introduce failures,
but be able to repeatedly introduce exactly the same failures in exactly the same way.
In the words of John Ousterhout,
"the only thing worse than a problem that happens all the time is a problem that doesn't happen all the time."
Therefore, our system must be \textit{deterministic}.

Furthermore,
distributed systems often run on comparatively long time-scales:
a system may be running for months or years,
hopefully without failure.
However,
it is unrealistic to spend months testing a system to confirm that such durations will not lead to failure.
But, with control of the machine's time system calls,
we can effectively "accelerate" time in the test,
meaning the distributed system to be tested will act as though it has been running for such a duration,
without interrupting development for such an extended period of time.

Finally,
distributed systems are highly varied,
and as a result may require features or platform support not foreseen by our implementation.
Modularity in our design allows our framework to be easily extended,
not future-proofing it but allowing it to be future-proofed.
}

\section{Related Work}

{\fontsize{12}{15}\selectfont 
Though not a research paper,
Redstone was inspired by work done by the TigerBeetle team,
and their work in testing the TigerBeetle database.
This work heavily involves hastened testing of simulated distributed system faults.
However, the limitation of their work is that their simulator is compiled specifically for their software.
While this setup works for their well-abstracted code that they can freely recompile to embed the simulation,
there are many, many more distributed systems in the real world that are a poor match for this form of simulation.
As such, we wanted to try to take the idea they built their replicator on.
The basis of their simulator involves mocking all input and output:
their work includes a network simulator to force network faults and latencies,
and an in-memory storage simulator to force storage faults and latencies.
The simulator then speeds up the clock to make these I/O operations occur far more frequently than in production,
causing faults that may only occur once a year, or even more infrequently, to occur in a matter of days.
Such functionality would be highly valuable in testing nearly any production distributed system,
but, again, the context of TigerBeetle’s work is highly specific.
As such, we wanted to develop a framework that incorporates a number of the basic simulations TigerBeetle’s simulator includes
(namely, hastened I/O faults for network and in-memory storage),
but one that can be generally implemented on normal systems and run a much broader range of applications.
}

% Distributed systems are fucking stupid.
% 
% Therefore, in order to combat this stupidity, there was the advent of a concept of "testing".
% This dark and arcane art is wonderfully usable but falls flat when tested conditions occur only
% rarely. The best example of such wild conditions are things like network latencies, disk
% io errors, kernel bugs, time, and more. Everything has some level of doing a fucky wucky,
% so our job is to make these fucky wuckies happen more often so that programmers are better able to
% understand how their system handles such fucky wuckies
% A paragraph of text goes here. Lots of text. Plenty of interesting
% text. Text text text text text text text text text text text text text
% text text text text text text text text text text text text text text
% text text text text text text text text text text text text text text
% text text text text text text text.
% More fascinating text. Features galore, plethora of promises.

% %-------------------------------------------------------------------------------
% \section{Footnotes, Verbatim, and Citations}
% %-------------------------------------------------------------------------------
% 
% Footnotes should be places after punctuation characters, without any
% spaces between said characters and footnotes, like so.%
% \footnote{Remember that USENIX format stopped using endnotes and is
% now using regular footnotes.} And some embedded literal code may
% look as follows.
% 
% \begin{verbatim}
% int main(int argc, char *argv[]) 
% {
% return 0;
% }
% \end{verbatim}
% 
% Now we're going to cite somebody. Watch for the cite tag. Here it
% comes. Arpachi-Dusseau and Arpachi-Dusseau co-authored an excellent OS
% book, which is also really funny~\cite{arpachiDusseau18:osbook}, and
% Waldspurger got into the SIGOPS hall-of-fame due to his seminal paper
% about resource management in the ESX hypervisor~\cite{waldspurger02}.
% 
% The tilde character (\~{}) in the tex source means a non-breaking
% space. This way, your reference will always be attached to the word
% that preceded it, instead of going to the next line.
% 
% And the 'cite' package sorts your citations by their numerical order
% of the corresponding references at the end of the paper, ridding you
% from the need to notice that, e.g, ``Waldspurger'' appears after
% ``Arpachi-Dusseau'' when sorting references
% alphabetically~\cite{waldspurger02,arpachiDusseau18:osbook}. 
% 
% It'd be nice and thoughtful of you to include a suitable link in each
% and every bibtex entry that you use in your submission, to allow
% reviewers (and other readers) to easily get to the cited work, as is
% done in all entries found in the References section of this document.
% 
% Now we're going take a look at Section~\ref{sec:figs}, but not before
% observing that refs to sections and citations and such are colored and
% clickable in the PDF because of the packages we've included.
% 
% %-------------------------------------------------------------------------------
% \section{Floating Figures and Lists}
% \label{sec:figs}
% %-------------------------------------------------------------------------------


% %---------------------------
% \begin{figure}
% \begin{center}
% \begin{tikzpicture}
% \draw[thin,gray!40] (-2,-2) grid (2,2);
% \draw[<->] (-2,0)--(2,0) node[right]{$x$};
% \draw[<->] (0,-2)--(0,2) node[above]{$y$};
% \draw[line width=2pt,blue,-stealth](0,0)--(1,1)
% node[anchor=south west]{$\boldsymbol{u}$};
% \draw[line width=2pt,red,-stealth](0,0)--(-1,-1)
% node[anchor=north east]{$\boldsymbol{-u}$};
% \end{tikzpicture}
% \end{center}
% \caption{\label{fig:vectors} Text size inside figure should be as big as
% caption's text. Text size inside figure should be as big as
% caption's text. Text size inside figure should be as big as
% caption's text. Text size inside figure should be as big as
% caption's text. Text size inside figure should be as big as
% caption's text. }
% \end{figure}
% %% %---------------------------


% Here's a typical reference to a floating figure:
% Figure~\ref{fig:vectors}. Floats should usually be placed where latex
% wants then. Figure\ref{fig:vectors} is centered, and has a caption
% that instructs you to make sure that the size of the text within the
% figures that you use is as big as (or bigger than) the size of the
% text in the caption of the figures. Please do. Really.
% 
% In our case, we've explicitly drawn the figure inlined in latex, to
% allow this tex file to cleanly compile. But usually, your figures will
% reside in some file.pdf, and you'd include them in your document
% with, say, \textbackslash{}includegraphics.
% 
% Lists are sometimes quite handy. If you want to itemize things, feel
% free:
% 
% \begin{description}
% 
% \item[fread] a function that reads from a \texttt{stream} into the
% array \texttt{ptr} at most \texttt{nobj} objects of size
% \texttt{size}, returning returns the number of objects read.
% 
% \item[Fred] a person's name, e.g., there once was a dude named Fred
% who separated usenix.sty from this file to allow for easy
% inclusion.
% \end{description}
% 
% \noindent
% The noindent at the start of this paragraph in its tex version makes
% it clear that it's a continuation of the preceding paragraph, as
% opposed to a new paragraph in its own right.
% 
% 
% \subsection{LaTeX-ing Your TeX File}
% %-----------------------------------
% 
% People often use \texttt{pdflatex} these days for creating pdf-s from
% tex files via the shell. And \texttt{bibtex}, of course. Works for us.

%-------------------------------------------------------------------------------
\section*{Acknowledgments}
%-------------------------------------------------------------------------------

The USENIX latex style is old and very tired, which is why
there's no \textbackslash{}acks command for you to use when
acknowledging. Sorry.

%-------------------------------------------------------------------------------
\section*{Availability}
%-------------------------------------------------------------------------------

USENIX program committees give extra points to submissions that are
backed by artifacts that are publicly available. If you made your code
or data available, it's worth mentioning this fact in a dedicated
section.

%-------------------------------------------------------------------------------
\bibliographystyle{plain}
\bibliography{\jobname}

%%%%%%%%%%%%%%%%%%%%%%%%%%%%%%%%%%%%%%%%%%%%%%%%%%%%%%%%%%%%%%%%%%%%%%%%%%%%%%%%
\end{document}
%%%%%%%%%%%%%%%%%%%%%%%%%%%%%%%%%%%%%%%%%%%%%%%%%%%%%%%%%%%%%%%%%%%%%%%%%%%%%%%%

%%  LocalWords:  endnotes includegraphics fread ptr nobj noindent
%%  LocalWords:  pdflatex acks
